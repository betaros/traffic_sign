\hypertarget{haar_cascades_dataset}{}\section{Haar Cascaden}\label{haar_cascades_dataset}
\hypertarget{haar_cascades_create_dataset}{}\subsection{Datensatz erstellen}\label{haar_cascades_create_dataset}
Für die Erstellung des Datensatzes zu Verkehrszeichenerkennung wurden Videoaufnahmen von den Verkehrszeichen aus mehreren Perspektiven gemacht. Danach wurde das Video mit ffmpeg in einzelne Bilder gespeichert.


\begin{DoxyCode}
ffmpeg -i videodatei bildname -hidebanner
\end{DoxyCode}


Dabei ist zu beachten, dann beim Bildname \%04d im Dateinamen mit angegeben wird, sodass die Bild vierstellig nummeriert werden. Die Anzahl an Ziffern kann, wenn nötig, angepasst werden z.\+B. \%06d.\hypertarget{haar_cascades_create_cascades}{}\subsection{Cascaden erstellen}\label{haar_cascades_create_cascades}
Der erstellte Datensatz muss nun in einer Liste mit Beschreibung zusammengefasst werden. Dabei wird der Pfad, die Anzahl an Eigenschaften, sowie die Region of Interest (R\+OI) angegeben.

Info.\+dat erstellen 
\begin{DoxyCode}
/Dateipfad/Datei.jpg AnzahlEigenschaften ROI.x1 ROI.y1 ROI.x2 ROI.y2
\end{DoxyCode}


Zur Erstellung von Haar Cascaden werden zusätzlich auch Bilder benötigt, welche die Strukturen, die erkannt werden sollen nicht enthalten. Dazu wird ein weiterer Datensatz angelegt mit negativ Bildern. Diese werden dann in einer Datei bg.\+txt aufgelistet.

Bg.\+txt erstellen 
\begin{DoxyCode}
/Dateipfad/Datei.jpg
\end{DoxyCode}


Vector Datei erstellen 
\begin{DoxyCode}
opencv\_createsamples -info info.dat -num Anzahl Bilder -w Breite -h Höhe -vec Ausgabedatei.vec
\end{DoxyCode}


Cascade \mbox{\hyperlink{namespace_training}{Training}} 
\begin{DoxyCode}
opencv\_traincascade -data Pfad zum Speichern -vec Pfad zur Vektordatei -bg Negativ Bilder -numPos Anzahl
       positiver Bilder -numNeg Anzahl negativer Bilder -numStages Anzahl an Vertiefungen -w Breite -h Höhe
\end{DoxyCode}
 